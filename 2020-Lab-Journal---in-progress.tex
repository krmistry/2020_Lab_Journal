\PassOptionsToPackage{unicode=true}{hyperref} % options for packages loaded elsewhere
\PassOptionsToPackage{hyphens}{url}
%
\documentclass[]{article}
\usepackage{lmodern}
\usepackage{amssymb,amsmath}
\usepackage{ifxetex,ifluatex}
\usepackage{fixltx2e} % provides \textsubscript
\ifnum 0\ifxetex 1\fi\ifluatex 1\fi=0 % if pdftex
  \usepackage[T1]{fontenc}
  \usepackage[utf8]{inputenc}
  \usepackage{textcomp} % provides euro and other symbols
\else % if luatex or xelatex
  \usepackage{unicode-math}
  \defaultfontfeatures{Ligatures=TeX,Scale=MatchLowercase}
\fi
% use upquote if available, for straight quotes in verbatim environments
\IfFileExists{upquote.sty}{\usepackage{upquote}}{}
% use microtype if available
\IfFileExists{microtype.sty}{%
\usepackage[]{microtype}
\UseMicrotypeSet[protrusion]{basicmath} % disable protrusion for tt fonts
}{}
\IfFileExists{parskip.sty}{%
\usepackage{parskip}
}{% else
\setlength{\parindent}{0pt}
\setlength{\parskip}{6pt plus 2pt minus 1pt}
}
\usepackage{hyperref}
\hypersetup{
            pdftitle={2020 Lab Journal},
            pdfauthor={Kelly Mistry},
            pdfborder={0 0 0},
            breaklinks=true}
\urlstyle{same}  % don't use monospace font for urls
\usepackage[margin=1in]{geometry}
\usepackage{graphicx,grffile}
\makeatletter
\def\maxwidth{\ifdim\Gin@nat@width>\linewidth\linewidth\else\Gin@nat@width\fi}
\def\maxheight{\ifdim\Gin@nat@height>\textheight\textheight\else\Gin@nat@height\fi}
\makeatother
% Scale images if necessary, so that they will not overflow the page
% margins by default, and it is still possible to overwrite the defaults
% using explicit options in \includegraphics[width, height, ...]{}
\setkeys{Gin}{width=\maxwidth,height=\maxheight,keepaspectratio}
\setlength{\emergencystretch}{3em}  % prevent overfull lines
\providecommand{\tightlist}{%
  \setlength{\itemsep}{0pt}\setlength{\parskip}{0pt}}
\setcounter{secnumdepth}{0}
% Redefines (sub)paragraphs to behave more like sections
\ifx\paragraph\undefined\else
\let\oldparagraph\paragraph
\renewcommand{\paragraph}[1]{\oldparagraph{#1}\mbox{}}
\fi
\ifx\subparagraph\undefined\else
\let\oldsubparagraph\subparagraph
\renewcommand{\subparagraph}[1]{\oldsubparagraph{#1}\mbox{}}
\fi

% set default figure placement to htbp
\makeatletter
\def\fps@figure{htbp}
\makeatother


\title{2020 Lab Journal}
\author{Kelly Mistry}
\date{7/24/2020}

\begin{document}
\maketitle

\hypertarget{section}{%
\subsubsection{7/24/2020}\label{section}}

C++ introductory course from Codecademy is 61\% completed. Notes from
the first 5 sections (hello world, variables, conditionals \& logic,
loops and vectors):

\begin{itemize}
\tightlist
\item
  possible variables types are: int for integers, double for
  floating-point numbers, char for individual characters, string for
  sequence of characters, and bool for true/false
\item
  for loop format is a little different from R, with the counter
  declared, the bounds of the coutner and the interval of the counter in
  3 separate parts. Ex: for (int i = 0; i \textless{} vector.size();
  i++) \{operation\}. The i++ means the counter advances by 1.
\item
  if else is very similar to R in terms of syntax
\item
  using vectors requires adding the vector library at the top of the
  script with \#include , and the type of variable in the vector is
  declared when the vector is created. Ex: std::vector vector\_name =
  \{vector contents\}
\item
  to print both variables, plain text and have a line break: std::cout
  \textless{}\textless{} variable\_name \textless{}\textless{} ``text''
  \textless{}\textless{} ``\n'';
\end{itemize}

\textbf{Cheatsheets for all of the course contents are here:
\url{https://www.codecademy.com/learn/learn-c-plus-plus/modules/learn-cpp-hello-world/cheatsheet}}

\hypertarget{section-1}{%
\subsubsection{7/29/2020}\label{section-1}}

Notes from Functions section: * similar structure for R functions, but
parameter types and return value type needs to be explicitly define:
return\_value\_type function(parameter\_1\_type parameter\_1,
parameter\_2\_type parameter\_2) \{ function operation \}

\hypertarget{section-2}{%
\subsubsection{8/13/2020}\label{section-2}}

Working through Applied Time Series Analysis course materials. No issues
with the first section, on matrix manipulation in R.

One cool function I hadn't come across before is diag(), which creates a
matrix with zeros except on the diagonal, which is the values you put in
diag(); one value means all diagonals will be that value 1, 0, 0, 0, 1,
0, 0, 0, 1, or you can specify each value in the diagonal: 1, 0, 0, 0,
2, 0, 0, 0, 3.

\end{document}
